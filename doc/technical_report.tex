\documentclass[12pt]{article}
\usepackage{amsmath}
\usepackage{amssymb}
\usepackage{graphicx}
\usepackage{hyperref}
\usepackage[latin1]{inputenc}
\usepackage{listings}
\usepackage{xcolor}

\title{Progetto SISTEMI OPERATIVI 2022-2023}
\author{Simone Cappabianca - Mat: 5423306 \\  simone.cappabianca@edu.unifi.it}
\date{Dicembre 31, 2023}

\setlength{\parindent}{4em}
\setlength{\parskip}{1em}

\begin{document}
\maketitle
\newpage

\tableofcontents
\newpage

\section{Istruzioni per la compilazione e esecusione}
Tutti i comandi sottostanti vanno eseguti dalla directory del progetto. \\
Per l'installazione del progetto \`{e} necessario eseguire i seguenti comandi:
\begin{enumerate}
    \item \texttt{make all}
    \item \texttt{make install}
\end{enumerate}
Per disinstallare il progetto \`{e} necessario eseguire il comando:
\begin{enumerate}
    \item \texttt{make uninstall}
\end{enumerate}
Per eseguire il progetto \`{e} necessario aprire una prima shell ed eseguire il
comando con l'opzione di lancio che desideriamo (\texttt{NORMALE}/\texttt{ARTIFICIALE}):
\begin{enumerate}
    \item \texttt{./bin/ecu.out "OPZIONE-DI-LANCIO"}
\end{enumerate}
In una seconda shell \`{e} necessario eseguire il comando:
\begin{enumerate}
    \item \texttt{./bin/hmi\_output.out}
\end{enumerate}

\section{Sistema obiettivo}
Il progetto \`{e} stato sviluppato sulla distribuzione linux {\bf Ubuntu 22.04 LTS}.

\section{Elementi facoltativi}

\begin{tabular}{|p{0.1\textwidth}|p{0.4\textwidth}|p{0.2\textwidth}|p{0.4\textwidth}|}
    \hline
    \textbf{\#} & \textbf{Elemento Facoltativo} & \textbf{Realizzato (SI/NO)} & \textbf{Metodo o file principale}\\
    \hline
    \textbf{1} &
    Ad ogni accelerazione, c'\`{e} una probabilit\`{a} di $10^{-5}$ che l'acceleratore 
    fallisca. In tal caso, il componente throttle control invia un segnalealla 
    Central ECU per evidenziare tale evento, e la Central ECU avvia la procedura 
    di ARRESTO & \textbf{NO} &  \\
    \hline
    \textbf{2} & Componente "forward facing radar" & \textbf{NO} & \\
    \hline
    \textbf{3} &
    Quando si attiva l'interazione con park assist,la Central ECU sospende (o 
    rimuove) tutti i sensori e attuatori, tranne park assist e surround view 
    cameras. & \textbf{NO} & \\
    \hline
    \textbf{4} &
    Il componente Park assist non \`{e} generato all'avvio del Sistema, ma creato 
    dalla Central ECU al bisogno. & \textbf{SI} &  \\
    \hline
    \textbf{5} &
    Se il componente surround view cameras \`{e} implementato, park assist 
    trasmette a Central ECU anche i byte ricevuti da surround view cameras. &
    \textbf{NO} & \\
    \hline
\end{tabular}

\begin{flushleft}
    \begin{tabular}{|p{0.1\textwidth}|p{0.4\textwidth}|p{0.2\textwidth}|p{0.4\textwidth}|}
        \hline
        \textbf{6} & Componente "surround view cameras" & \textbf{NO} & \\
        \hline
        \textbf{7} &
        Il comando di PARCHEGGIO potrebbe arrivare mentre i vari attuatori stanno 
        eseguendo ulteriori comandi (accelerare o sterzare). I vari attuatori 
        interrompono le loro azioni, per avviare le procedure di parcheggio. &
        \textbf{NO} & \\
        \hline
        \textbf{8} &
        Se la Central ECU riceve il segnale di fallimento accelerazione da "throttle 
        control", imposta la velocit\`{a} a 0 e invia all'output della HMI un 
        messaggio di totale terminazione dell'esecuzione. & \textbf{NO} & \\
        \hline
    \end{tabular}    
\end{flushleft}

\section{Progettazione e implementazione}
Le scelte implementativi per la realizzazione del progetto sono le seguenti:
\begin{itemize}
    \item il componente {\bf Central ECU} di occupa di generare i processi dei 
    componenti necessari per esecuzione ad esclusione dell'output della Human-Machine 
    Interface;
    \item la {\bf Human-Machine Interface} \`{e} stata divisa in due processi 
    distinti uno relativo all'input e uno relativo all'output. 
\end{itemize}
Nello specifico il {\bf Central ECU} genera i processi dei seguenti componenti:
\begin{enumerate}
    \item {\bf front windshield camera};
    \item {\bf steer-by-wire};
    \item {\bf throttle control};
    \item {\bf brake-by-wire};
    \item {\bf Human-Machine interface} (input);
    \item {\bf park assit}.
\end{enumerate}
Mentre i processi dei primi 5 componeti vengono generati al momento dell'avvio 
della {\bf Central ECU}, il componente {\bf park assist} viene creato quando 
{\bf Central ECU} riceve il comando {\it PARCHEGGIO}. 

Per quando riguarda la comunicazione tra i processi \`{e} stato utilizzato un 
socket di tipo FIFO nei seguenti casi:
\begin{itemize}
    \item {\bf front windshield camera} $\to$ {\bf Central ECU};
    \item {\bf Central ECU} $\to$ {\bf steer-by-wire};
    \item {\bf Central ECU} $\to$ {\bf brake-by-wire};
    \item {\bf Central ECU} $\to$ {\bf throttle control};
    \item {\bf Central ECU} $\to$ {\bf Human-Machine Intervace output};
    \item {\bf park assist} $\to$ {\bf Central ECU}.
\end{itemize} 

Sono stati usati i segnali invece nei seguenti casi:
\begin{itemize}
    \item {\bf Human-Machine Interface input} $\to$ {\bf Central ECU} per la 
    gestione tutti i comandi in input ({\it INIZIO, PARCHEGGIO, ARRESTO});
    \item {\bf Central ECU} $\to$ {\bf brake-by-wire} per la gestione del comando 
    {\it ARRESTO}
    \item  {\bf Central ECU} $\to$ {\bf brake-by-wire} per la gestione del comando 
    {\it PERICOLO}
\end{itemize}
\section{Esecuzione}

\end{document}